\documentclass[11pt]{article}
\usepackage{fullpage}
\usepackage{amsthm}
\usepackage{amsthm,amsmath,amsfonts,amssymb,amstext}
\usepackage{latexsym,ifthen,url,rotating}
\usepackage[usenames,dvipsnames]{color}
\newtheorem{definition}{Definition}

\usepackage[letterpaper]{geometry}
\geometry{top=1in, bottom=1in, left=1in, right=1in}

\newcommand{\concat}{{\,\|\,}}
\newcommand{\bits}{\{0,1\}}
\newcommand{\Range}{{\mathrm{Range}}}
\newcommand{\A}{{\mathcal{A}}}

% Hi Paul,

% The first three problems don't require anything we've done recently:  
% They boil down to counting the number of possible outcomes that make the event happen.  
% Nothing fancy with random variables.

% For number 4, you need to solve for x. 
% Try computing the probabilities P(A), P(B), P(A intersect B), and using independence.

% For number 5, you really need to just go through what the random variables are doing on the sample space.  
% For number 6, you can figure out how to do the sample space, and then think about the definition of range, 
% partition, and frequency function with sample space and random variable.  I worked an example about 
% independence (like problem 6 part 2) in class.

% Problems 5 and 6 use notation from the notes I posted to Sakai.

% Hope that helps,
% Prof Cash

\begin{document}

\noindent Paul Jones \\
Rutgers University\\
CS206: Introduction to Discrete Structures II, Spring 2013\\
Professor David Cash\\

\begin{center}
Homework 4
\end{center}

\begin{enumerate}

	\item (3 points) In the card game bridge we deal 13-card hands to 4 players
		  named North, South, East, and West (all 52 cards are dealt).  What is the
		  probability that East and West have no spades?
	
	\begin{itemize}
	
		\item The total number of possibilities in the sample space for this situation is
			  the total number of possible card orders over the number of cards each player
			  is dealt factorial raised to the fourth.
		
			  \[ \frac{52!}{13!^4} \]
		
		\item There are 13 Spades in a deck of cards, and then 39 non-Spade cards.
		\item For the hands of East and West, who collectively represent 26 cards,
			  and who are vying for 39 non-Spade cards, this is the number of 
			  possibilities:
		
			  \[ 39 \choose 26 \]
		
		\item North and South also account for 26 total cards, and they are vying
			  for the remaining cards including Spades. Twenty-six cards were taken
			  out of the deck by East and West.
			  
		     \[ {26 \choose 26} = 1 \]
		     
		\item The probability is therefore:
		
		     \[ \frac{{39 \choose 13}\times{26 \choose 13}\times{26 \choose 13}\times{13 \choose 13}}{\frac{52!}{13!^4}} \]
		     
	\end{itemize}
	
	\item (4 points) An urn contains $n>0$ white balls and $m>0$ black balls.
		  Suppose we draw two balls without replacement.  What is the probability that
		  the balls are of the same color?  What if we draw them with replacement?  Show
		  your work.  Which of these probabilities is larger?  Briefly explain some
		  intuition for why one should be larger.
		  
		  \begin{itemize}
		  
			  \item Without replacement
			  
			  \begin{itemize}
			  
				  \item There are $n + m$ balls in all, and we're interested in choosing
				  		2 of them. The sample space is:
						
						\[ {m + n \choose 2} \]
						
				  \item Call $W$ ``drawing two white balls'' and $B$ ``drawing two black
				  		balls. Then $W \cup B$ is either of those events happening or both.
						
				  \item The events are mutually exclusive, the drawings cannot all be the same
				  		for the same drawing.
						
						\[ P(W \cup B) = P(W) + P(B) \]
						\[ = \frac{{m \choose 2} + {n \choose 2}}{{m + n \choose 2}} \]
						
			  \end{itemize}
			  
			  \item With replacement
			  
			  \begin{itemize}
			  
			      \item The events are, again, mutually exclusive.
			      \item Being as the balls are placed back in the urn, each event is equally
			      	    likely to occur.
				  \item Count the number of either color, place it over total balls, and
				        multiply it by itself to represent that the event has to happen twice.
				  \item Do the same for the other color, add the two values.
			      
			      \[ \left(\frac{m}{m + n}\right)^2 + \left(\frac{n}{m + n}\right)^2 \]
			  
			  \end{itemize}
		  
		  \end{itemize}
	
	\item (4 points) Again consider an urn with $n>0$ white balls and $m >0$ black
		  balls.  Suppose we draw $r\geq 1$ balls from the urn without replacement.  What
		  is the probability that we draw exactly $k$ white balls?
	
	\begin{itemize}
	
		\item The probability of getting a white, where $i$ is the number of total balls already drawn, is
		
			  \[ \left(\frac{n - i}{m + n - i}\right) \]
			  
		\item And the probability of getting a black under the same conditions is
		
			  \[ \left(\frac{m - i}{m + n - i}\right) \]
			  
		\item So draw $k$ white balls, and then transfer the counter to drawing the 
			  remaining balls to avoid over-counting
			  
			  \[ \left(\prod_{i = 0}^{k - 1} \frac{n - i}{m + n - i} \right) \times \left(\prod_{i = k}^{r - 1} \frac{m - (i - k)}{m + n - i} \right)\]
	
	\end{itemize}
	
	\item (4 points) A box contains a mixture of cubes and spheres and any of these
		  objects can be either white or black.  Suppose the box contains $4$ black
		  cubes, $6$ black spheres, $6$ white cubes, and $x$ white spheres.  Consider the
		  experiment of drawing a random object from the box, and let $A$ be the event
		  that a cube is drawn and $B$ be the event that a black object is drawn.
		  If $A$ and $B$ are independent, what is $x$?
	
	\begin{itemize}
	
		\item The probability of a cube being drawn is the total number of cubes over
			  the total number of cubes and spheres.
			  
			  \[ \frac{4 + 6}{4 + 6 + 6 + x} \]
			  
		\item The probability of a black object being drawn is similarly the total number
			  of black objects divided by the total number of objects
			  
			  \[\frac{6 + 4}{4 + 6 + 6 + x} \]
	
		\item Consider the complement of $A \cup B$, it is the event that a non-black
			  and non-cube item is draw, which is a white sphere.
			  
			  \[P(x) = P((A \cup B)^c) \]
	
	\end{itemize}
	
	\item (10 points total) Two fair dice are rolled.  Define the random variables
		  $X =$ the sum of the two rolls,
		  $Y =$ the maximum of the two rolls,
		  $Z =$ the absolute value of the difference of the two rolls
		  and $W = XY$ (i.e., the product of $X$ and $Y$).
	\begin{enumerate}
	 \item (2 points) What are $\Range(X)$, $\Range(Y)$, $\Range(Z)$ and
	 $\Range(W)$?
	 
	 \begin{itemize}
	 
		 \item $\Range(X) = \lbrace 2, 3, 4, ... , 11, 12 \rbrace $
		 \item $\Range(Y) = \lbrace 1, 2, 3, 4, 5, 6 \rbrace $
		 \item $\Range(Z) = \lbrace 0, 1, 2, 3, 4, 5 \rbrace$
		 \item $\Range(W) = \lbrace i \times j \: : \: i \in \lbrace 2, 3, 4, ... , 11, 12 \rbrace, j \in \lbrace 1, 2, 3, 4, 5, 6 \rbrace\rbrace $
	 
	 \end{itemize}
	
	 \item (2 points) What are the partitions $\A_X$ and $\A_Z$?
	 
	 \begin{itemize}
	 
	 	\item $\A_X$
		
			\begin{itemize}
			
				\item $2 = 1 + 1$
				\item $3 = 1 + 2$
				\item $4 = 1 + 3 = 2 + 2$
				\item $5 = 1 + 4 = 3 + 2$
				\item $6 = 1 + 5 = 2 + 4 = 3 + 3$
				\item $7 = 1 + 6 = 2 + 5 = 3 + 4$
				\item $8 = 2 + 6 = 3 + 5 = 4 + 4$
				\item $9 = 3 + 6 = 4 + 5$
				\item $10 = 4 + 6 = 5 + 5$
				\item $11 = 5 + 6$
				\item $12 = 6 + 6$
			
			\end{itemize}
	 
	 	\item $\A_Z$
		
			\begin{itemize}
			
				\item $0 = 1 - 1 = 2 - 2 = 3 - 3 = 4 - 4 = 5 - 5 = 6 - 6$
				\item $1 = 6 - 5 = 5 - 4 = 4 - 3 = 3 - 2 = 3 - 2 = 2 - 1$
				\item $2 = 5 - 4 = 5 - 3 = 4 - 2 = 3 - 1$
				\item $3 = 6 - 3 = 5 - 2 = 4 - 1$
				\item $4 = 6 - 2 = 5 - 1$
				\item $5 = 6 - 1$
			
			\end{itemize}
	 
	 \end{itemize}
	 
	 \item (3 points) Give tables showing the values of $f_X,f_Y,f_Z,$ and $f_W$.
	 
	 
		
		\begin{table}[h!]
			\begin{center}
			\caption{$f_X$}
				\begin{tabular}{|c|c|c|c|c|c|c|c|c|c|c|c|}
				\hline
				x & 2 & 3 & 4 & 5 & 6 & 7 & 8 & 9 & 10 & 11 & 12 \\ 
				\hline
				f(x) & $\frac{1}{11}$ & $\frac{1}{11}$ & $\frac{2}{11}$ & $\frac{2}{11}$ & $\frac{3}{11}$ & $\frac{3}{11}$ & $\frac{3}{11}$ & $\frac{2}{11}$ &$\frac{2}{11}$ & $\frac{1}{11}$ & $\frac{1}{11}$ \\ 
				\hline
				\end{tabular}
			\label{ }
			\end{center}
		\end{table} 
		
		\begin{table}[h!]
			\begin{center}
			\caption{$f_Y$}
				\begin{tabular}{|c|c|c|c|c|c|c|}
				\hline
				x & 1 & 2 & 3 & 4 & 5 & 6\\ 
				\hline
				f(y) & $\frac{1}{6}$ & $\frac{1}{6}$ & $\frac{2}{6}$ & $\frac{3}{6}$ & $\frac{4}{6}$ & $\frac{5}{6}$ \\ 
				\hline
				\end{tabular}
			\label{ }
			\end{center}
		\end{table} 
		
		\begin{table}[h!]
			\begin{center}
			\caption{$f_Z$}
				\begin{tabular}{|c|c|c|c|c|c|c|}
				\hline
				x & 0 & 1 & 2 & 3 & 4 & 6\\ 
				\hline
				f(z) & $\frac{6}{12}$ & $\frac{5}{12}$ & $\frac{4}{11}$ & $\frac{3}{12}$ & $\frac{2}{12}$ & $\frac{1}{12}$ \\ 
				\hline
				\end{tabular}
			\label{ }
			\end{center}
		\end{table} 
	 
	 \item (3 points) Are the events $X =7$ and $Z=1$ independent?
	 
	 \begin{itemize}
	 
	 	\item No. It doesn't meet the definition.
	 
	 \end{itemize}
	 
	\end{enumerate}
\end{enumerate}


\end{document}

