


\documentclass[11pt]{article}
\usepackage{fullpage}
\usepackage{amsthm}
\usepackage{amsthm,amsmath,amsfonts,amssymb,amstext}
\usepackage{latexsym,ifthen,url,rotating}
\usepackage[usenames,dvipsnames]{color}


% --- -----------------------------------------------------------------
% --- Document-specific definitions.
% --- -----------------------------------------------------------------
\newtheorem{definition}{Definition}

\newcommand{\concat}{{\,\|\,}}
\newcommand{\bits}{\{0,1\}}
\newcommand{\Range}{{\mathrm{Range}}}
\newcommand{\A}{{\mathcal{A}}}

% --- -----------------------------------------------------------------
% --- The document starts here.
% --- -----------------------------------------------------------------
\begin{document}
%\maketitle
\sloppy

\noindent Paul Jones \\
Rutgers University\\
CS206: Introduction to Discrete Structures II, Spring 2013\\
Professor David Cash\\

\begin{center}
Homework 4
\end{center}

\begin{enumerate}


\item (10 points total) Two teams $x$ and $y$ are playing each other in the
World Series, which is a best-of-seven-game match that ends when one
team wins 4 games.  Assume that team $x$ wins each game with probability
$p$, and that the outcome of each game constitutes an independent trial.

\begin{enumerate}
	\item (0.5 points) What is the probability that $x$ wins the first four games?
	
	\begin{itemize}
	
		\item There is one way for x to win in four, and that's winning four in a row.
		\item By the multiplicity principle, we can multiply the chance of winning each game to get the probability of winning all games.
		
			  \[p^4\]
	
	\end{itemize}
	
	\item (2 points) What is the probability that $x$ wins four games after
	at most five game have been played?
	
	\begin{itemize}
		\item Sample space, strings, x character means x won, y character means y won.
		
		\begin{verbatim}
		
		xxxx
		yxxxx
		xyxxx
		xxyxx
		xxxyx
		
		\end{verbatim}
		
		\item The team represented by y cannot win at the end because after 4 x wins, the games halt.
		\item P(x wins in less than five games) = $P(\mathrm{\{xxxx, yxxxx, xyxxx, xxyxx, xxxyx\}})$
		\item There is a $(1 - p)$ chance of y winning.
		\item The x team still needs to win 4 games, which has a probability of $p^4$.
		\item Multiplying these two values gets you the likelihood that this happens for any given instance of x winning in less than five games.
		\item To get all of the possible less than 5 games combinations, multiply by the number there are, which is 4.
		\item Then, add the probability of just 4 straight victories.
		\[(1 - p) \times p^4 \times 4 + p^4\]
		%\item P(x wins in 6) = ${5 \choose 2}(1 - p)^2 p^4$
		
	\end{itemize}
	
	\item (2 points) What is the probability that $x$ will win four games before
	$y$ wins four games?  (i.e., What is the probability that $x$ wins
	the Series?)
	
	\begin {itemize}
	
		\item To begin, lets enumerate some possibilities using strings.
		
		\[ \{\mathrm{xxxx}\} \]
		
		\[\{\mathrm{yxxxx,
		xyxxx,
		xxyxx,
		xxxyx}\}\]
		
		\[\{\mathrm{yyxxxx,
		yxyxxx,
		...,
		xxxyyx}\}\]
		
		\[\{\mathrm{yyyxxxx,
		yyxyxxx,
		...,
		xxxyyyx}\}\]
		
		\item These are all the possibilities for x winning. Name these $X_1$ through $X_4$ and 
			  notice the sum of their probabilities to be the probability that $x$ wins four games.
			  
		\item For $X_3$, there are ${5 \choose 2,3}$ ways of ordering x and y, y has to win twice, and x still has to win four times.
		\item You have to subtract off the cases where there is a y at the end, 
		
			  \[ {4 \choose 1,3} \times (1 - y)^2 \times p^4 \]
	
		\item For $X_4$, there are ${6 \choose 3,3}$ ways of ordering x and y, y has to win thrice, and x still has to win four times.
		
			  \[ {5 \choose 2,3} \times (1 - y)^3 \times p^4 \]
			  
		\item Now add 4 straight victories:
		
			  \[ {4 \choose 1,3} \times (1 - y)^2 \times p^4 + {5 \choose 2,3} \times (1 - y)^3 \times p^4 + p^4\]
	\end{itemize}
	
	\item (0.5 points) Calculate and simplify your answer in part (c) when $p=1/2$
	and when $p=2/3$.
	
	\begin{itemize}
	
		\item $p = 1 / 2 : 1/2$
		\item $p = 2 / 3 : 1808/2187$
	
	\end{itemize}
	
	\item (1 point) Let $X$ be the random variable that counts the number of games
	that are played.  What is $\Range(X)$?
	
	\[R(X) = \{4, 5, 6, 7 \} \]
	
	\item (2 points) What is $P(X=7)$?
	
	\[ {6 \choose 3}(1 - p)^3 p^4 + {6 \choose 3} p^3 (1 - p)^3 p^4 \]
	
	\item (2 points) What is $P(X\geq 6)$?
	
	\[ {5 \choose 2}(1 - p)^2 p^4 + {5 \choose 2} p^2 (1 - p)^4 + {6 \choose 3} (1 - p)^3 p^4 + {6 \choose 3} p^3 (1 - p)^4 \]
	
\end{enumerate}

\item (4 points) Suppose we roll two fair dice.  Let the random variable $X=$
``the minimum of the two dice'' and $Y =$ ``the absolute value of the
difference of the two dice''.  Find $E(X)$ and $E(Y)$.

\begin{itemize}

	\item $E(X)$
	
	\begin{itemize}
	
		\item The formula to solve this is
		
			  \[E(X) = \sum_{a_i \in R(X)}a_i P(X = a_1)\]
			
		\item The sample space has a cardinality of 36, as there 6 choices for each of the two choices.
		\item All rolls are equally likely in a fair dice.
		\item Starting with the highest element in the range (which is the set containing 1 through 6), there is only one way 6 can be the minimum.
	
			  \[P(x = 6) = P(\lbrace(6, 6) \rbrace) = \frac{1}{36}\]
			  
		\item There are 3 ways of ``getting 5'', and that's rolling two fives, and then both variations of a five and a six.
		
			  \[P(x = 5) = P(\lbrace(5, 5), (5, 6), (6,5) \rbrace) = \frac{3}{36}\]
			  
		\item Apply the same pattern,
	
			  \[P(x = 4) = P(\lbrace(4, 4), (4, 5), (5,4), (4,6), (6,4) \rbrace) = \frac{5}{36}\]
			  \[P(x = 3) = P(\lbrace(3, 3), (3, 4), (4,3), (3,5), (5,3), (3,6), (6,3) \rbrace) = \frac{7}{36}\]
			  \[P(x = 2) = P(\lbrace(2, 2), (2, 3), (3,2), (2, 4), (4, 2), (2, 5), (5,2), (2,6), (6,2)) = \frac{9}{36} \]
			  
		\item Notice that there 2 more each time.
		
			  \[P(x = 1) = \frac{11}{36} \]
			  
		\item Now sum them and multiply them by their value ($a_i$) to find expected value,
		
			  \[E(X) = \left(6 \times \frac{1}{36}\right) + \left(5 \times \frac{3}{36}\right) + \left(4 \times \frac{5}{36}\right) + \left(3 \times \frac{7}{36}\right) + \left(2 \times \frac{9}{36}\right) + \frac{11}{36} = 2.5277777778\]
			  
	\end{itemize}
	
	\item $E(Y)$

	\begin{itemize}
	
		\item The range is $\{0, 1, 2, 3, 4, 5\}$.
		\item This is another application of the formula
			  \[E(X) = \sum_{a_i \in R(X)}a_i P(X = a_1)\]
		\item There are 6 ways of getting 0, $\{6 - 6, 5 - 5, ...\}$. For $i = 0$,
		      \[ 0 \times P(X = 0) = 0 \times \frac{6}{36} = 0 \]
		\item There are 10 ways of getting 1, $\{6 - 5, 5 - 6, 5 - 4, 4 - 5, 4 - 3, 3 - 4, 3 - 2, 2 - 3, 2 - 1, 1 - 2\}$. For $i = 1$,
		      \[ 1 \times P(X = 1) = 1 \times \frac{10}{36} \]
		\item There are 8 ways of getting 2, $\{6 - 4, 4 - 6, 5 - 3, 3 - 5, 4 - 2, 2 - 4, 1 - 5, 5 - 1\}$. For $i = 2$,
			  \[ 2 \times P(X = 2) = 2 \times \frac{8}{36} = \frac{4}{9}\]
		\item There are 6 ways of getting 3, $\{6 - 3, 3 - 6, 5 - 2, 2 - 5, 4 - 1, 1 - 4\}$. For $i = 3$,
			  \[ 3 \times P(X = 3) = 3 \times \frac{6}{36} = \frac{2}{3}\]
	    \item There are 4 ways of getting 4, $\{6 - 2, 2 - 6, 5 - 1, 1 - 5\}$. For $i = 4$,
	          \[ 4 \times P(X = 4) = 4 \times \frac{4}{36} = \frac{4}{9}\]
	    \item There are 2 ways of getting 5, $\{6 - 1, 1 - 6\}$. For $i = 5$,
	          \[ 5 \times P(X = 5) = 5 \times \frac{2}{36} = \frac{5}{18}\]
	    \item We know we've covered the sample space because the sum of the specific instances is the same as the cardinality of the sample space.
	    \item The expected value,
	    		  \[E(X) = 0 + \frac{2}{9} + \frac{4}{9} + \frac{2}{3} + \frac{4}{9} +  \frac{5}{18} = 1.94 \approx 2\]
	
	\end{itemize}

\end{itemize}

\item (4 points) Suppose boxes of cereal are filled with a random prize,
each drawn from independently and uniformly from $6$ possible prizes.
If we buy $N$ boxes of cereal, what is the expected number of distinct
prizes we will collect? \begin{small}\textsf{Hint: Consider the indicator
random variables $I_{E_i}$ for the event $E_i =$ ``the $i$-th price was
in some box''.}\end{small}

\begin{itemize}

	\item Let $X_1 ... X_6$ be the identifier for each of 6 unique toys
	
	\[ E\left(\sum I_{E_i}\right) = \sum_{i = 1}^6 E(I_{E_i}) = 6(1 - \left(5/6)\right)^n \]

\end{itemize}

\item (4 points) A group of $m$ men and $w$ randomly sit in a single row at a
theater.  If a man and woman are seated next to each other we say they form a
couple.  (Couples can overlap, meaning that one person can be a member of two
couples.)  What is the expected number of couples?
\begin{small}\textsf{Hint: Use indicator
random variables for each possible couple forming.
}\end{small}

\begin{itemize}

	\item Use linearity of expectation for each pair of seats.
	\item Let $x$ equal the number of seats.
	\item Then, $E(X_1) = P($couple in a seat 1 and 2$)$, $E(X_2) = P($couple in a seat 2 and 3$)$, etc.
	\item For any given seat, there is a $\frac{1}{2}$ chance of their being a man or a woman in the seat.
	\item The four possibilities are \{mw, wm, mm, ww\}.
		  \[\sum_{i}^{x} \]

\end{itemize}

\item (3 points) Suppose an experiment tosses a fair coin twice;  the experiment
``succeeds'' if both tosses were Heads.  We repeat this experiment 
for 12 independent trials.  Let $N$ be the random variable that counts
the fraction of trials that are successful (so $N = S/12$, where
$S$ is the number of successful trials).  Find $E(N)$.


	\[ E(N) = E\left(\frac{S}{12}\right) \]
	\[ S = \{ x_1 + ... + x_12 \} \]
	\[\frac{1}{12} E(S) \]
	\[E(S) = \sum_{a_i \in R(S)} a_i \times P(X_i) \]
	\[P(X_i) = \frac{1}{4} \]
	\[E(S) = 12 \times E(X_i) \]
	\[\frac{1}{12}E(S) = 12 \times \frac{1}{4} \]
	\[ E(S) = \frac{1}{4} \]


\item \textbf{Extra Credit: (4 points)} Consider the experiment where $n$ balls
are to be placed randomly into $n$ boxes. Let $N_1$ count the number of boxes
with exactly one ball, and let $N_2$ count the number of boxes with exactly two
balls. Find the probability of the events ``$N_1 = n$" and ``$N_1 = n - 1$".
Use the indicator technique to find $E(N_1)$ and $E(N_2)$.


\end{enumerate}


\end{document}

