


\documentclass[11pt]{article}
\usepackage{fullpage}
\usepackage{amsthm}
\usepackage{amsthm,amsmath,amsfonts,amssymb,amstext}
\usepackage{latexsym,ifthen,url,rotating}
\usepackage[usenames,dvipsnames]{color}


% --- -----------------------------------------------------------------
% --- Document-specific definitions.
% --- -----------------------------------------------------------------
\newtheorem{definition}{Definition}

\newcommand{\concat}{{\,\|\,}}
\newcommand{\bits}{\{0,1\}}
\newcommand{\Range}{{\mathrm{Range}}}
\newcommand{\A}{{\mathcal{A}}}

% --- -----------------------------------------------------------------
% --- The document starts here.
% --- -----------------------------------------------------------------
\begin{document}
%\maketitle
\sloppy

\noindent Paul Jones \\
Rutgers University\\
CS206: Introduction to Discrete Structures II, Spring 2013\\
Professor David Cash\\

\begin{center}
Homework 4
\end{center}

\begin{enumerate}


\item (10 points total) Two teams $x$ and $y$ are playing each other in the
World Series, which is a best-of-seven-game match that ends when one
team wins 4 games.  Assume that team $x$ wins each game with probability
$p$, and that the outcome of each game constitutes an independent trial.

\begin{enumerate}
	\item (0.5 points) What is the probability that $x$ wins the first four games?
	
	\begin{itemize}
	
		\item There is one way for x to win in four, and that's winning four in a row.
		
			  \[p^4\]
	
	\end{itemize}
	
	\item (2 points) What is the probability that $x$ wins four games after
	at most five game have been played?
	
	\begin{itemize}
		\item Sample space, strings, x character means x won, y character means y won.
		
		\begin{verbatim}
		
		xxxx
		yxxxx
		xyxxx
		xxyxx
		xxxyx
		
		\end{verbatim}
		
		\[(1 - p) p^4 \times 4 \]
		
		\item P(x wins in less than five games) = $P(\mathrm{\{xxxx, yxxxx, xyxxx, xxyxx, xxxyx\}})$
		\item P(x wins in 6) = ${5 \choose 2}(1 - p)^2 p^4$
		
	\end{itemize}
	
	\item (2 points) What is the probability that $x$ will win four games before
	$y$ wins four games?  (i.e., What is the probability that $x$ wins
	the Series?)
	
	
	
	\item (0.5 points) Calculate and simplify your answer in part (c) when $p=1/2$
	and when $p=2/3$.
	
	\item (1 point) Let $X$ be the random variable that counts the number of games
	that are played.  What is $\Range(X)$?
	
	\item (2 points) What is $P(X=7)$?
	
	\item (2 points) What is $P(X\geq 6)$?
\end{enumerate}

\item (4 points) Suppose we roll two fair dice.  Let the random variable $X=$
``the minimum of the two dice'' and $Y =$ ``the absolute value of the
difference of the two dice''.  Find $E(X)$ and $E(Y)$.

\begin{itemize}

	\item $E(X)$
	
	\begin{itemize}
	
		\item 
	
	\end{itemize}
	
	\item $E(Y)$

	\begin{itemize}
	
		\item The range is $\{0, 1, 2, 3, 4, 5\}$.
		\item There are 6 ways of getting 0, $\{6 - 6, 5 - 5, ...\}$
		      \[ \frac{6}{36} \]
		\item There are 6 ways of getting 1, $\{6 - 5, 5 - 6, 4 - 3, 3 - 4, 3 - 2, 2 - 3, 2 - 1, 1 - 2\}$.
		      \[ \frac{8}{36} \]
	
	\end{itemize}

\end{itemize}

\item (4 points) Suppose boxes of cereal are filled with a random prize,
each drawn from independently and uniformly from $6$ possible prizes.
If we buy $N$ boxes of cereal, what is the expected number of distinct
prizes we will collect? \begin{small}\textsf{Hint: Consider the indicator
random variables $I_{E_i}$ for the event $E_i =$ ``the $i$-th price was
in some box''.}\end{small}

\item (4 points) A group of $m$ men and $w$ randomly sit in a single row at a
theater.  If a man and woman are seated next to each other we say they form a
couple.  (Couples can overlap, meaning that one person can be a member of two
couples.)  What is the expected number of couples?
\begin{small}\textsf{Hint: Use indicator
random variables for each possible couple forming.
}\end{small}


\item (3 points) Suppose an experiment tosses a fair coin twice;  the experiment
``succeeds'' if both tosses were Heads.  We repeat this experiment 
for 12 independent trials.  Let $N$ be the random variable that counts
the fraction of trials that are successful (so $N = S/12$, where
$S$ is the number of successful trials).  Find $E(N)$.

\item \textbf{Extra Credit: (4 points)} Consider the experiment where $n$ balls
are to be placed randomly into $n$ boxes. Let $N_1$ count the number of boxes
with exactly one ball, and let $N_2$ count the number of boxes with exactly two
balls. Find the probability of the events ``$N_1 = n$" and ``$N_1 = n - 1$".
Use the indicator technique to find $E(N_1)$ and $E(N_2)$.


\end{enumerate}


\end{document}

