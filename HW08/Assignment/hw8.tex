


\documentclass[11pt]{article}
\usepackage{fullpage}
\usepackage{amsthm}
\usepackage{amsthm,amsmath,amsfonts,amssymb,amstext}
\usepackage{latexsym,ifthen,url,rotating}
\usepackage[usenames,dvipsnames]{color}


% --- -----------------------------------------------------------------
% --- Document-specific definitions.
% --- -----------------------------------------------------------------
\newtheorem{definition}{Definition}

\newcommand{\concat}{{\,\|\,}}
\newcommand{\bits}{\{0,1\}}
\newcommand{\Range}{{\mathrm{Range}}}
\newcommand{\A}{{\mathcal{A}}}

% --- -----------------------------------------------------------------
% --- The document starts here.
% --- -----------------------------------------------------------------
\begin{document}
%\maketitle
\sloppy

\noindent Rutgers University\\
CS206: Introduction to Discrete Structures II, Spring 2013\\
Professor David Cash\\

\begin{center}
\LARGE{\textbf{Homework 8}}\\
\large{\textbf{\emph{Due at the beginning of class on Monday, April 22}}}
\end{center}

\vspace{.1in}

\noindent\textbf{Instructions:} Point values for each problem are listed.
Write your solutions neatly or type them up.  Typed solutions will also be
accepted via Sakai.

\begin{enumerate}

\item (7 points each) Solve the following recurrences using generating
functions.  Show all of your work.  Solutions without detailed
explanations will receive little or no credit.
\begin{enumerate}
\item $a_0 = 1, a_1=1$, and for $n\geq 1$, $a_n = a_{n-1} + 2a_{n-2}$
\item $a_0 = 1, a_1=1$, and for $n\geq 2$, $a_n = a_{n-1} + 2 a_{n-2} + 4$
\item $a_0 = 1$, and for $n\geq 1$, $a_n = 3a_{n-1} + 4^{n-1}$
\end{enumerate}

\item \textbf{(Extra credit: 5 points)} Generating functions can also
be used to prove some difficult identities.  Prove that
\[
\binom{a+b}{k} = 
\sum_{i=0}^n \binom{a}{i}\binom{b}{k-i} 
\]
by finding the generating function that has the left-hand side as its
$k$-th coefficient, and then showing 
that it is equal to the product of two generating
functions and applying the convolution formula.  (See Example 5 in
the scanned notes for a similar example.)
\end{enumerate}


\end{document}

