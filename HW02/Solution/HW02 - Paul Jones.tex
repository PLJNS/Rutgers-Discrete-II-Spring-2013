\documentclass[11pt]{article}
\usepackage{fullpage}
\usepackage{amsthm,amsmath,amsfonts,amssymb,amstext}

\begin{document}

{
\noindent Paul Jones\\
Professor David Cash  \\
Introduction to Discrete Structures II (01:198:206) \\
\begin{center}
Homework 1
\end{center}
}

\begin{enumerate}

\item (8 points) Prove that $P(A \cup B) \le P(A) + P(B)$ for any events A and B. Prove the general version by induction, which says that if $A_1, ..., A_n$ are events then 
$P(\bigcup_{i = 1}^n A_i) \le P(\sum_{i = 1}^n A_i)$. When does this inequality become an equality?

\begin{itemize}

\item Base case 

\begin{equation*}
P(A_1 \cup A_2) \le P(A_1) + P(A_2)
\end{equation*}

\item Assume it holds for a general case $(n=k)$
\begin{equation*}
P\left(\bigcup_{i = 1}^k A_i\right) \le \sum_{i = 1}^k P(A_i)
\end{equation*}

\begin{equation*}
P(A_1 \cup A_2 \cup ... \cup A_k) \le P(A_1) + P(A_2) + ... + P(A_k)
\end{equation*}

\item Prove that it hold for next case using your assumption.

\begin{equation*}
P\left(\bigcup_{i = 1}^{k+1} A_i\right) \le \sum_{i = 1}^{k+1} P(A_i)
\end{equation*}

\begin{equation*}
P(A_1 \cup A_2 \cup ... \cup A_k \cup A_{k+1}) \le P(A_1) + P(A_2) + ... + P(A_k) + P(A_{k+1})
\end{equation*}

\begin{equation*}
(A_1 \cup A_2 \cup ... \cup A_k) = C
\end{equation*}

\begin{equation*}
 P(A_1) + P(A_2) + ... + P(A_k) = D
\end{equation*}

\begin{equation*}
P(C \cup A_{k+1}) \le P(D) + P(A_{k+1})
\end{equation*}

By the base case, this is true, and expression is proven for all events.

\end{itemize}



\item (4 points) If $P(A) = \frac{1}{2}$, $P(B) = \frac{1}{5}$, and $P(A\cup B) = 3/5$, what are $P(A\cap B)$, $P(A^c \cup B)$, and $P(A^c \cap B)$?

\begin{itemize}

\item $P(A\cap B) = \frac{1}{10}$
\item $P(A^c \cup B) = 3/5$ because the probability of $A$ is the same as $A^c$ 
\item $P(A^c \cap B) = \frac{1}{10}$

\end{itemize}


\item (3 points) How many elements are there in the set

\{$x : 10^7 \le x \le 10^8$, and the base 10 representation of x has no digit used twice\}?

\begin{itemize}
\item $10^7 = 10000000$, and $10^8 = 100000000$
\item The smallest number possible is $12345678$, the greatest number possible is $98765432$
\item For the first number in any element, the first option is \{1, 2, 3, 4, 5, 6, 7, 8, 9\}.
\item Then for the every second number, it's every number besides the first choice plus 0.
\item The ``tree'' looks as follows, where $i\neq j\neq k\neq l\neq m\neq n\neq o\neq p$:
\begin{enumerate}
\item $\{1 i j k l m n o \}$ where $\{i,j,k,l,m,n,o\} \in \{0,2,3,4,5,6,7,8,9\}$
\item $\{2 i j k l m n o \}$ where $\{i,j,k,l,m,n,o\} \in \{0,1,3,4,5,6,7,8,9\}$
\item $\{3 i j k l m n o \}$ where $\{i,j,k,l,m,n,o\} \in \{0,1,2,4,5,6,7,8,9\}$
\item $\{4 i j k l m n o \}$ where $\{i,j,k,l,m,n,o\} \in \{0,1,2,3,5,6,7,8,9\}$
\item $\{5 i j k l m n o \}$ where $\{i,j,k,l,m,n,o\} \in \{0,1,2,3,4,6,7,8,9\}$
\item $\{6 i j k l m n o \}$ where $\{i,j,k,l,m,n,o\} \in \{0,1,2,3,4,5,7,8,9\}$
\item $\{7 i j k l m n o \}$ where $\{i,j,k,l,m,n,o\} \in \{0,1,2,3,4,5,6,8,9\}$
\item $\{8 i j k l m n o \}$ where $\{i,j,k,l,m,n,o\} \in \{0,1,2,3,4,5,6,7,9\}$
\item $\{9 i j k l m n o \}$ where $\{i,j,k,l,m,n,o\} \in \{0,1,2,3,4,5,6,7,8\}$
\end{enumerate} 
\item Every ``node'' (a) through (i) is $9 \choose 1$.
\item When you pick the first number, you still have 9 choices because zero is added. When you choose the second number, you have 8 choices because a number is taken out and non are put back into the set of choices. When you pick your third number, you have 7 choices because a number is taken out and none put back in. ...
\item So for integer strings $i$ through $o$:
\begin{equation*}
[i]\quad\quad[j]\quad\quad[k]\quad\quad[l]\quad\quad[m]\quad\quad[n]\quad\quad[o]\quad\quad[p]
\end{equation*}

\item The number of elements is equal to

\begin{equation*}
{9 \choose 1}\times{9 \choose 1}\times{8 \choose 1}\times{7 \choose 1}\times{6 \choose 1}\times{5 \choose 1}\times{4 \choose 1}\times{3 \choose 1}
\end{equation*}

\end{itemize}


\item (3 points) An army output has 19 posts to staff using 30 indistinguishable guards. How many ways are there to distribute the guards if no post is left empty?

\begin{itemize}
\item This is a ``stars and bars'' problem, where the number of ``stars'' $k$ is equal to 30, and the number of ``bars,'' ``bins,'' or ``posts'' $n$ is equal to 19.
\begin{equation*}
{k - 1 \choose n - 1} = {30 - 1 \choose 19 - 1} = {29 \choose 18} = 34597290
\end{equation*}
\end{itemize}



\item (1 point) What is the coefficient of $x^{10}y^{13}$ when $(x + y)^{23}$ is expanded?

\begin{itemize}

\item This problem requires the binomial theorem, where $n = 23$:

\begin{equation*}
(x + y)^{23} = \sum_{k = 0}^{23} {23 \choose k}x^ky^{23 - k}
\end{equation*}

\item When $k = 10$, $23 - k$ will equal 13.

\begin{equation*}
{23 \choose 10}x^{10} y^{13}
\end{equation*}

\item So the coefficient for $x^{10} y^{13}$ will be

\begin{equation*}
{23 \choose 10} = 1144066
\end{equation*}

\end{itemize}



\item (4 points) What is the coefficient of $w^{9}x^{31}y^{4}z^{19}$ when $(w + x + y + z)^{63}$ is expanded? How
many monomials appear in the expansion?

\begin{itemize}

\item This is a multinomial coefficient problem where $k = 4$ and $n = 63$:

\begin{equation*}
(a_1+a_2+...+a_k)^n=\sum_{\substack{n_1,n_2,...,n_k\ge 0 \\ 
n_1+n_2+...+n_k=n}}\frac{n!}{n_1!n_2!...n_k!}a_1^{n_1}a_2^{n_2}...a_k^{n_k}
\end{equation*}

\item Set $n_1 = 9$, $n_2 =31$, $n_3 = 4$, and $n_4 = 19$.

\begin{equation*}
\left(\frac{63!}{9!31!4!19!}\right)\times w^{9}x^{31}y^{4}z^{19}
\end{equation*}


\item Therefore, the coefficient is
\begin{equation*}
\frac{63!}{9!31!4!19!}
\end{equation*}

\item The number of monomials in a multinomial coefficient can be expressed as a ``stars and bars'' problem.

\begin{equation*}
66 \choose 3
\end{equation*}

\end{itemize}



\item (6 points) Let $p$ be a prime number and $1 \le k \le p - 1$. Prove that $p \choose k$ is a multiple of $p$.
Show that this is not true if $p$ is not prime.

\begin{equation*}
{p\choose k} = \frac{p(p-1) \times ... \times (p - k + 1)}{1 \times 2 \times ... \times (k - 1) \times k} = p{p - 1 \choose k}
\end{equation*}

\begin{equation*}
{4\choose 2} = 6
\end{equation*}

6 is not a multiple of 4 and is not prime.
 


\item Verify that for any $n \ge k \ge 1$
\begin{equation*}
{n \choose 2} = {k \choose 2} + k(n - k) + {n - k \choose 2}
\end{equation*}
Then give a combinatorial argument for why this is true.

\begin{itemize}
\item Observe that the binomial coefficient with two for all reals:
\begin{equation*}
{r\choose 2} = \frac{r (r - 1)}{ 2}
\end{equation*}
\item Apply fact to both sides:
\begin{equation*}
\frac{n (n - 1)}{ 2} = \frac{k (k - 1)}{ 2} + k(n - k) + \frac{(n - k) (n - k - 1)}{ 2}
\end{equation*}
\item Multiply everything by two
\begin{equation*}
n (n - 1) = k (k - 1) + 2k(n - k) + (n - k)(n - k - 1)
\end{equation*}
\item Expand out
\begin{equation*}
n (n - 1) = k^2 - k + 2kn - 2k^2 + k+k^2-n-2 k n+n^2
\end{equation*}
\item Simplify
\begin{equation*}
n (n - 1) = -n+n^2
\end{equation*}
\item Change form
\begin{equation*}
n (n - 1) = n (n - 1)
\end{equation*}

\item Combinatorial argument: 
\begin{itemize}
\item $n \choose 2$ is the number of ways we can arrange $n$ objects into groups of 2.
\item Splitting $n$ into 2, one of those groups will be of size $k$, making the other set of size $n - k$
\item Using the new partitions $k$ and $n-k$, arrange $n$ ``things'' into groups of two.
\item The first partition is of length $k$, resulting in $k\choose 2$ ways, and for the second part, we have $n-k$ choose 2 ways, explaining the first and last terms. 
\item Now choose 2 ``things'', one from different groups of 2, we can choose 1 of the k objects and all of $n-k$ ``things'', which is equal to $k(n-k)$.
\end{itemize}
\end{itemize}
\end{enumerate}


\end{document}

