


\documentclass[11pt]{article}
\usepackage{fullpage}
\usepackage{amsthm}
\usepackage{amsthm,amsmath,amsfonts,amssymb,amstext}
\usepackage{latexsym,ifthen,url,rotating}
\usepackage[usenames,dvipsnames]{color}


% --- -----------------------------------------------------------------
% --- Document-specific definitions.
% --- -----------------------------------------------------------------
\newtheorem{definition}{Definition}

\newcommand{\concat}{{\,\|\,}}
\newcommand{\bits}{\{0,1\}}

% --- -----------------------------------------------------------------
% --- The document starts here.
% --- -----------------------------------------------------------------
\begin{document}
%\maketitle
\sloppy

\noindent Rutgers University\\
CS206: Introduction to Discrete Structures II, Spring 2013\\
Professor David Cash\\

\begin{center}
\LARGE{\textbf{Homework 3}}\\
\large{\textbf{\emph{Due at the beginning of class on Wednesday, Feb 27}}}
\end{center}

\vspace{.1in}

\noindent\textbf{Instructions:} Point values for each problem are listed.
Write your solutions neatly or type them up.  Typed solutions will also be
accepted via Sakai.

\begin{enumerate}

\item (6 points) What the probability that a 5 card hand contains exactly 3
spades?  What if we condition on the hand containing at least 1 spade?

\item (7 points) Suppose $n$ people each throw a six-sided die.  Let $A_n$ be
the event that at least two distinct people roll the same number.  Calculate
$P(A_n)$ for $n=1,2,3,4,5,6,7$.

\item (3 points) Suppose we draw 2 balls at random from an urn that contains 5 distinct
balls, each with a different number from $\{1,2,3,4,5\}$, and define the events
$A$ and $B$ as
\[
A = \text{``5 is drawn at least once"} \quad \text{and} \quad
B = \text{``5 is drawn twice"} 
\]
Compute $P(A)$ and $P(B)$.

\item (4 points) In the previous problem, suppose we place the first ball back
in the urn before drawing the second.  Compute $P(A),P(B),P(A|B),P(B|A)$ in
this version of the experiment.


\item (4 points) Suppose $5$ percent of cyclists cheat by using illegal doping.
The blood test for doping returns positive $98$ percent of the people
doping and $12$ percent who do not.  If Lance's test comes back positive,
what the probability that he is doping? (Ignoring all other evidence,
of course...)

\item (6 points) If $A \subseteq B$, can $A$ and $B$ be independent?  What we
if require that $P(A)$ and $P(B)$ both not equal $0$ or $1$?

\item \textbf{Extra credit (5 points)}  Consider the experiment where two dice
are thrown. Let $A$ be the event that the sum of the two dice is 7. For each 
$i \in \{1,2,3,4,5,6\}$  let $B_i$ be the event that at least one $i$ is thrown.
\begin{enumerate}
\item Compute $P(A)$ and $P(A|B_1)$.
\item Prove that $P(A|B_i) = P(A|B_j)$ for all $i$ and $j$.
\item Since you know that some $B_i$ always occurs, does it make sense that
$P(A) \neq P(A | B_i)$?  (After all, if $E$ is an event with $P(E) = 1$,
then for any event $F$, $P(F|E) = P(F)$.  What is going on?  Does this
seem paradoxical?)
\end{enumerate}

\end{enumerate}


\end{document}

