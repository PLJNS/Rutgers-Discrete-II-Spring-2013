\documentclass[11pt]{article}
\usepackage{fullpage}
\usepackage{amsthm}
\usepackage{amsthm,amsmath,amsfonts,amssymb,amstext}
\usepackage{latexsym,ifthen,url,rotating}
\usepackage[usenames,dvipsnames]{color}


% --- -----------------------------------------------------------------
% --- Document-specific definitions.
% --- -----------------------------------------------------------------
\newtheorem{definition}{Definition}

\newcommand{\concat}{{\,\|\,}}
\newcommand{\bits}{\{0,1\}}

% --- -----------------------------------------------------------------
% --- The document starts here.
% --- -----------------------------------------------------------------
\begin{document}
%\maketitle
\sloppy

\noindent Paul Jones \\
Rutgers University\\
CS206: Introduction to Discrete Structures II, Spring 2013\\
Professor David Cash\\

\begin{center}
Homework 3 
\end{center}

\vspace{.1in}

\begin{enumerate}

\item (6 points) What the probability that a 5 card hand contains exactly 3
spades?  What if we condition on the hand containing at least 1 spade?

\begin{itemize}

\item A deck of cards has 52 cards, 4 suits, 13 ranks, one of each suit/rank combination.
\item There are $52 \choose 5$ possible hands.
\item There are $13$ cards which are spades.
\item Let $E_i = $ ``a spade was drawn'' and $F_i = $ ``anything except a spade was drawn.''
\item We want $P(E_1 \cap E_2 \cap E_3 \cap F_1 \cap F_2)$.
\item The probability of $E_1$ is $\frac{13}{52}$, because thirteen of the cards are spades.
\item Assuming $E_1$, the probability of $E_2$ is $\frac{12}{51}$, because there is one less spade and one less card in general.
\item Assuming $E_2$, the probability of $E_3$ is $\frac{11}{50}$, because there are now two less spades and two less cards in general.
\item There are now 49 cards in all, 10 of which \emph{are} spades. 
\item Assuming $E_1$ through $E_3$, the probability of $F_1 = \frac{39}{49}$.
\item Assuming $F_1$, the probability of $F_2 = \frac{38}{48}$.
\begin{equation*}
\frac{13}{52}\times\frac{12}{51}\times\frac{11}{50}\times\frac{39}{49}\times\frac{38}{48} = 0.008154261704
\end{equation*}

\item Alternatively, there are 13 choose 3 ways of picking a spade, 39 choose 2 way of picking a ``not spade,'' and there are 52 choose 5 possible options:

\begin{equation*}
\frac{{13 \choose 3} {39 \choose 2}}{{52 \choose 5}} = \frac{{286} \times {741}}{{2598960}} = 0.008154261704
\end{equation*}

\item For part two, you only have to pick four cards out of 51, because one is a space already.

\begin{equation*}
\frac{{13 \choose 2} {39 \choose 2}}{{51 \choose 4}} 
\end{equation*}

\end{itemize}

\clearpage

\item (7 points) Suppose $n$ people each throw a six-sided die.  Let $A_n$ be
the event that at least two distinct people roll the same number.  Calculate
$P(A_n)$ for $n=1,2,3,4,5,6,7$.

\begin{itemize}

\item For $A_1$, It is impossible for two distinct people to roll the same number if only person rolls.
\[P(A_1) = 0\]
\item For $A_2$, there are 36 possible throws, but only 6 of could contain two distinct people rolling the same number.
\[P(A_2) = \frac{6}{6^2} = \frac{1}{6} = 0.1666666667\]
\item For $A_3$ consider the complement, which can be described as ``No two distinct people roll the same number.'' 
The probability that all three people roll unique numbers is $1 \times \frac{5}{6} \times \frac{4}{6} \times$, 
\[P(A_3) = 1 - P(A_3^c) = 1 - \frac{5}{6} \times \frac{4}{6} = 0.4444444444\]
\item For $A_4$, consider the complement again, and apply the same reasoning.
\[P(A_4) = 1 - 1 \times \frac{5}{6} \times \frac{4}{6} \times \frac{3}{6} = \frac{13}{18} =0.7222222222 \]
\item For $A_5$, consider the complement again, and apply the same reasoning.
\[P(A_5) = 1 - \frac{5}{6} \times \frac{4}{6} \times \frac{3}{6} \times \frac{2}{6} = 0.9074074074\]
\item For $A_6$, consider the complement again.
There are $6!$ ways of arranging the integer elements in the string ``123456.'' This is out of a $6^6$ ways of writing
a string with integers from one to six.
\[P(A_6) = 1 - \frac{6!}{6^6} = 0.98456790123\]
\item If 7 people roll a dice with 6 sides, it is inevitable that at least two distinct people roll the same number.
\[P(A_7) = 1\]

\clearpage

%\item There are only 6 sides on this die, so if at least 7 people roll die, then it's inevitable that at least 2 people rolled the same number.
%\item Therefore, $P(A_7) = 1$.
%\item Similarly, if only one person rolls one die, then it's impossible that two people rolled the same number.
%\item Therefore, $P(A_1) = 0$. But $A_1$ can roll any number.
%\item Now, consider what it would take for each roll to be unique, which is the complement of the desired event.
%\item $A_1^c = 1$
%\item $A_1$ rolled some number, and the probability that $A_2^c$ rolls a different number is $\frac{5}{6}$.
%\item For $A_3^c$, the probability of a different roll is $\frac{4}{6}$, because 2 numbers have been ``taken.''
%\item For $A_4^c$, the probability of a different roll is $\frac{3}{6}$, because 3 numbers have been ``taken.''
%\item For $A_5^c$, the probability of a different roll is $\frac{2}{6}$, because 4 numbers have been ``taken.''
%\item For $A_6^c$, the probability of a different roll is $\frac{1}{6}$, because 5 numbers have been ``taken.''
%\item For $A_7^c$, the probability is zero, all numbers have been taken.
%\item For $A_{6}^c$, the overall probability is $1 \times \frac{5}{6} \times \frac{4}{6} \times \frac{3}{6} \times \frac{2}{6} \times \frac{1}{6} = 0.01543209877$
%\item Which means for $A_6$, the probability is $1 - 0.01543209877 = 0.9845679012$.
%\item Applying this same reasoning, $P(A_5) = 1 - \frac{5}{6} \times \frac{4}{6} \times \frac{3}{6} \times \frac{2}{6} = 0.9074074074$.
%\item $P(A_4) = 1 - \frac{5}{6} \times \frac{4}{6} \times \frac{3}{6} =0.7222222222$
%\item $P(A_3) = 1 - \frac{5}{6} \times \frac{4}{6} = 0.4444444444$
%\item $P(A_2) = 1 - \frac{5}{6} =0.1666666667$
%\item $P(A_1) = 1 - 1 = 0$

\end{itemize}

\item (3 points) Suppose we draw 2 balls at random from an urn that contains 5 distinct
balls, each with a different number from $\{1,2,3,4,5\}$, and define the events
$A$ and $B$ as
\[
A = \text{``5 is drawn at least once"} \quad \text{and} \quad
B = \text{``5 is drawn twice"} 
\]
Compute $P(A)$ and $P(B)$.

\begin{itemize}

\item $P(B) = 0$
\item $P(A) = \frac{4}{{5\choose 2}} = 0.4$

\end{itemize}

\item (4 points) In the previous problem, suppose we place the first ball back
in the urn before drawing the second.  Compute $P(A),P(B),P(A|B),P(B|A)$ in
this version of the experiment.

\begin{itemize}

\item $P(A) = 1 - \left(\frac{4}{5}\right)^2 = .36 $
\item $P(B) = \left(\frac{1}{5}\right)^2 = .04$
\item $P(A|B) = 1$
\item $P(B|A) = \frac{1}{5}$

\end{itemize}

\item (4 points) Suppose $5$ percent of cyclists cheat by using illegal doping.
The blood test for doping returns positive $98$ percent of the people
doping and $12$ percent who do not.  If Lance's test comes back positive,
what the probability that he is doping? (Ignoring all other evidence,
of course...)

\begin{itemize}

	\item $P(C) = $ ``the probability a cyclist cheated by doping'' $= .05$
	\item $P(T|C) = $ ``the probability a cyclist giving a positive test if they doped'' $= .98$
	\item $P(T|C^c) = $ ``the probability a cyclist giving a positive test if they \emph{did not} dope'' $ = .12$
	\item We want the probability of a cyclist doping given a positive test, which is $P(C|T)$.

	\begin{equation*}
		P(T|C) = .98 = \frac{P(T \cap C)}{P(C)} = \frac{P(T \cap C)}{.05}
	\end{equation*}
	
	\begin{equation*}
		P(T \cap C) = .049
	\end{equation*}
	
	\begin{equation*}
	P(T \cap C^c) = .12 = \frac{P(T \cap C^c)}{P(C^c)} = \frac{P(T \cap C^c)}{.95}
	\end{equation*}
	
	\begin{equation*}
		P(T \cap C^c) = .114
	\end{equation*}
	
	\item For any event $T$ and $C$,
	
	\begin{equation*}
		P(T) = P(T \cap C) + P(T \cap C^c) 
	\end{equation*}
	
	\begin{equation*}
		P(T) = .049 + .114 = .163
	\end{equation*}
	
	\begin{equation*}
		P(C|T) = \frac{P(C \cap T)}{P(T)} = \frac{.049}{.163} = 0.3006134969 \approx .30
	\end{equation*}
	
	\item Intuition:
	
	\begin{itemize}
	
		\item Take 100 cyclists, 5 of them actually doped according to these numbers.
		\item If all 100 cyclists are tested, it's very, very likely the 5 will return positive.
		\item Of the remaining 95 cyclists, 12 percent of them will also return positive, which makes a little less than 12 cyclists.
		\item So for each of the 17 who tested positive, there are five who actually doped, which means each has a $\frac{5}{17} = 0.2941176471 \approx .30$ probability of doping.
	
	\end{itemize}

\end{itemize}

\item (6 points) If $A \subseteq B$, can $A$ and $B$ be independent?  What we
if require that $P(A)$ and $P(B)$ both not equal $0$ or $1$?

\begin{itemize}

\item $A$ and $B$ are independent if they satisfy the condition $P(A \cap B) = P(A)P(B)$
\item If $A \subseteq B$, it is true that $A \cap B = B$.
\item This means that $P(A \cap B)=P(B)$.
\item In order to be independent when $A \subseteq B$ is true, $P(B)$ must equal $P(A)\cdot P(B)$.
\item The only way anything multiplied by something can equal itself is if that something is one.
\item Therefore, when $A \subseteq B$, $A$ and $B$ can be independent when $P(A) = 1$.
\item Furthermore, being as anything multiplied by zero yields zero, when $A \subset B$, $A$ and $B$ can be independent if $P(B) = 0$.
\item So no.

\end{itemize}

\item \textbf{Extra credit (5 points)}  Consider the experiment where two dice
are thrown. Let $A$ be the event that the sum of the two dice is 7. For each 
$i \in \{1,2,3,4,5,6\}$  let $B_i$ be the event that at least one $i$ is thrown.
\begin{enumerate}
\item Compute $P(A)$ and $P(A|B_1)$.

\begin{itemize}

\item $P(A) = \frac{6}{6^2} = \frac{1}{6}$
\item $P(A|B_1) = \frac{P(A \cap B_1)}{P(B_1)}= \frac{\frac{3}{{6 \choose 2}}}{\frac{1}{6} + \frac{1}{6}} = 0.6$

\end{itemize}

\item Prove that $P(A|B_i) = P(A|B_j)$ for all $i$ and $j$.
\begin{itemize}
\item Being as the sum has to be seven, there is one and only way to sum to seven for the integers 1 through 6.
\item So it doesn't matter what is rolled on the first roll, the probability ``rides on'' the second roll being the number the first roll needs to sum to seven.
\item The probability of rolling any given number on a fair die is always $\frac{1}{6}$.
\item Therefore, for all $i$ and $j$, the probability cannot be anything but $\frac{1}{6}$.
\end{itemize}

\item Since you know that some $B_i$ always occurs, does it make sense that
$P(A) \neq P(A | B_i)$?  (After all, if $E$ is an event with $P(E) = 1$,
then for any event $F$, $P(F|E) = P(F)$.  What is going on?  Does this
seem paradoxical?)
\end{enumerate}

\end{enumerate}


\end{document}
